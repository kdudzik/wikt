Żyjemy w~czasach, w~których nieustannie zmienia się sposób wyszukiwania informacji przez przeciętnego człowieka. Z~roku na rok coraz mniejszą rolę odgrywają papierowe kompendia takie jak encyklopedie i~słowniki, stopniowo przybierają natomiast na znaczeniu elektroniczne bazy wiedzy --- szczególnie zaś internetowe zbiory danych. Przyczyny tego stanu rzeczy są oczywiste: chodzi przede wszystkim o~wygodę korzystania ze~stron internetowych. Brak możliwości wyszukiwania w~obrębie ogromnych ilości danych powoduje, że encyklopedie i~słowniki w~postaci książek stają się o~wiele mniej atrakcyjne dla kogoś, kto chce zdobyć nowe informacje.

Wszechobecny dostęp do internetu sprawia, że to właśnie w~sieci WWW powstają najbardziej popularne bazy ludzkie wiedzy. Nie ma chyba internautów, którzy nie korzystaliby, rzadziej lub częściej, z~Wikipedii --- internetowej encyklopedii pisanej przez ochotników. Właśnie fakt, że encyklopedia ta współtworzona jest przez amatorów, stanowi o~jej wyjątkowym charakterze, który zostanie w~tej pracy pokrótce opisany. Wikipedia stale utrzymuje się w~pierwszej dziesiątce najczęściej odwiedzanych stron, a~pod wieloma względami jest to dziś najlepsza istniejąca encyklopedia. Przed kilkoma laty głośne było porównanie jej z~prestiżową \emph{Encyclopædia Britannica} --- okazało się, że różnice w~poziomie merytorycznym są niewielkie.

O~ile przewrót w~kategorii encyklopedii właściwie już się dokonał, nieco inaczej wygląda rywalizacja słowników. Oczywiście wyraźnie widać, że i~tu papierowe edycje są coraz mniej popularne. Różnice uwidaczniają się, gdy przeanalizowana zostanie sytuacja słowników internetowych. Tak zwany siostrzany projekt Wikipedii, Wikisłownik, nie dominuje wśród konkurencji --- zarówno na świecie, jak i~w~Polsce. Przyczyny tego stanu rzeczy są złożone. Autor postanowił skupić się na kilku zagadnieniach, uwidaczniających się w~polskojęzycznej wersji Wikisłownika. W~tym celu konieczne było zbadanie społeczności zaangażowanej w~tworzenie tego projektu. Jego efektem było wykonanie prac programistycznych, których opis stanowi główną część niniejszego opracowania.

W~przypadku wszystkich projektów opartych na silniku programistycznym MediaWiki istotną barierą rozwoju jest sama technologia. Każdy ochotnik ma możliwość uczestniczenia w~rozwoju portalu, wiąże się to jednak z~koniecznością przystosowania się do wymagań stawianych przez oprogramowanie. Edytowanie haseł w~internetowej encyklopedii czy słowniku jest praktycznie niemożliwe dla osoby bez wcześniejszego przygotowania lub znacznej wiedzy techniczno\dywiz{}informatycznej. Oprogramowanie MediaWiki oparte jest bowiem na tzw. wikikodzie (także: wikitekst, wikiskładnia), czyli języku opisu struktury i~wyglądu strony internetowej --- prostszym niż HTML, jednak wciąż nieintuicyjnym dla kogoś, kto nie miał wcześniej do czynienia z~tego typu edytorami. Dlatego wielu potencjalnych współautorów zniechęca się do projektu już przy pierwszej próbie poprawy artykułu.

Aby zmienić tę sytuację, przygotowany został nowy edytor, dostosowany specjalnie do potrzeb polskiego Wikisłownika. Aplikacja pozwala na o~wiele prostsze tworzenie nowych i~zmienianie starych haseł niż poprzednia, standardowa. Dzięki użyciu jej jako domyślnej w~projekcie popularyzacja edytowania Wikisłownika wśród fachowców w~dziedzinach lingwistycznych okaże się łatwiejsze --- zniknie podstawowa bariera, jaką jest konieczność dostosowania się do skomplikowanych technicznych wymagań stawianych przez użyte oprogramowanie. Dodatkowo nowa aplikacja umożliwia zaawansowaną automatyzację tworzenia hasła. Wiele z~czynności zintegrowanych z~nowym edytorem do tej pory wymagało mozolnych poszukiwań w~artykułach Wikisłownika oraz innych projektach. Dzięki użyciu API udostępnianego przez serwisy Fundacji Wikimedia skomplikowane przeszukiwanie tysięcy stron udało się sprowadzić do kilku kliknięć.

W~dalszej części pracy opisany został proces tworzenia tego edytora. Pierwszy rozdział charakteryzuje pokrótce sam Wikisłownik, jak i~pokrewne projekty oraz oprogramowanie w~nich użyte. Następnie opisano społecznościowe aspekty tworzenia tego typu aplikacji ze szczególnym uwzględnieniem koncepcji \emph{wiki}. Ostatni rozdział wyczerpująco przedstawia szczegóły projektowe i~implementacyjne aplikacji.
