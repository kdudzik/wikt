\section{Koncepcja \emph{wiki}}
\section{Społeczność polskiej edycji Wikisłownika}
\label{sec:plsoc}


\begin{table}[h]
\begin{center}
	\begin{tabularx}{\textwidth}{ XX }
		\toprule \textbf{Wikipedia} & \textbf{Wikisłownik} \\
		\midrule Dużo wandalizmów (edycji wykonywanych w~złej wierze)
			& Bardzo mało wandalizmów \\
		\midrule Kilkuset stale aktywnych redaktorów
			& Kilkunastu stale aktywnych redaktorów \\
		\midrule Wielu użytkowników mających kłopoty z~dostosowaniem się do zasad
			& Niewielu użytkowników mających kłopoty z~dostosowaniem się do zasad \\
		\midrule Liczne wojny edycyjne (wzajemne cofanie swoich edycji przez co najmniej dwóch redaktorów)
			& Praktycznie brak wojen edycyjnych \\
		\bottomrule
	\end{tabularx}
\caption
	[Porównanie Wikipedii i~Wikisłownika -- aspekty społeczne]
	{Porównanie Wikipedii i~Wikisłownika -- aspekty społeczne. Zob. też tabelę \ref{tab:wiki-wikt}}
\label{tab:wiki-wikt2}
\end{center}
\end{table}

Można uznać, że w~przypadku polskich wersji językowych Wikisłownik jest projektem znacznie spokojniejszym niż Wikipedia, której powszechność powoduje nieuniknione problemy, takie jak zaangażowanie osób chcących wykorzystać projekt do celów marketingowych lub ideologicznych czy liczne wandalizmy w~wykonaniu znudzonych nastolatków. Społeczność redaktorów słownika zdecydowanie skupiona jest na stałym udoskonalaniu projektu. Z~racji jego rozmiaru łatwiej jest o~konsensus w~dyskusjach, a~dzięki mniejszej popularności innowacyjne rozwiązania techniczne mają większą szansę na realizację.

\section{Analiza wymagań}

\section{Specyfika tworzenia aplikacji dla wikispołeczności}

