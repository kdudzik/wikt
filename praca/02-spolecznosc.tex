Tematem tego rozdziału są społecznościowe aspekty zarówno samego Wikisłownika, jak i~procesu powstawania niniejszej pracy magisterskiej. Na początku omówiono podstawowe założenia koncepcji \emph{wiki}, na której opiera się Wikisłownik. Kolejna część rozdziału to krótka analiza społeczności rozwijającej projekt. Na dalszych stronach znalazły się treści ściśle związane z~tworzoną aplikacją: opis rozwoju oprogramowania dla specyficznego klienta, jakim jest szersza społeczność, oraz szczegółowa analiza jego wymagań.

\section{Koncepcja \emph{wiki}}
Pojęcie \emph{wiki} określa witrynę internetową umożliwiającą tworzenie i~edycję dowolnej liczby połączonych ze sobą hiperłączami stron przez przeglądarkę internetową, za pomocą uproszczonego języka znaczników lub edytora WYSIWYG \cite{britannica}. Opisane w~rozdziale~\ref{chap:wikt} projekty Fundacji Wikimedia, szczególnie Wikipedia, są z~pewnością najbardziej znanymi i~największymi przykładami wiki.

Wiki to nie tylko technologia, ale także --- a~może przede wszystkim --- społeczność. Rozmiary wiki rozciągają się od małych, zamkniętych witryn firmowych do największej Wikipedii. To, co łączy wszystkie tego typu projekty, to otwarty dostęp do edycji i~tworzenia stron w~obrębie danej grupy. W~przypadku projektów Fundacji grupę tę stanowi niemal cała ludność świata dysponująca dostępem do internetu --- poza nielicznymi krajami, w~których sieć jest cenzurowana. Wśród redaktorów Wikipedii czy Wikisłownika nie istnieje żadna skodyfikowana hierarchia, co oznacza, że niezalogowany nowicjusz może zmienić stronę dokładnie tak samo jak administrator projektu. Założenie to stanowi od lat główny zarzut pod adresem Wikipedii --- negatywne opinie o~tej encyklopedii są szeroko rozpowszechnione i~chętnie cytowane przez jej przeciwników~\cite{knol}. Mimo tego projekty stale rozwijają się i~podnoszą swoją jakość --- już w~2005~roku w~czasopiśmie ,,Nature'' opublikowano kontrowersyjne porównanie Wikipedii z~\emph{Encyclopædia Britannica}, z~którego wynikało, że obie encyklopedie stoją na podobnym poziomie~\cite{nature:britannica}. Jasno widać, że znaczenie projektów Fundacji Wikimedia jest dziś bardzo duże.

W~tworzeniu Wikipedii i~Wikisłownika biorą udział przedstawiciele całego społeczeństwa --- w~pracach uczestniczą zarówno profesorowie, jak i~studenci. Jak wspomniano w~podrozdziale~\ref{wikt:drawbacks}, dla wielu z~nich barierą jest poziom skomplikowania oprogramowania MediaWiki. Podczas gdy wielu uczestników, najczęściej tych o~zainteresowaniach ścisłych, nie ma problemów z~opanowaniem techniki edycji haseł, istnieje duża grupa osób, które deklarują, że mogłyby zaangażować się bardziej, gdyby nie konieczność opanowania złożonych sposobów wprowadzania informacji. Dzięki regularnej strukturze akurat polski Wikisłownik jest tym projektem, w~którym uproszczenie tworzenia i~edycji haseł jest najbardziej możliwe do przeprowadzenia. Ponieważ proces taki wymaga wspólnej decyzji ogółu społeczności, należało zanalizować, jak kształtują się interakcje między użytkownikami w~projekcie.

\section{Społeczność polskiej edycji Wikisłownika}
\label{sec:plsoc}
W~punkcie~\ref{subs:wiki-wikt} przedstawione zostały różnice między technicznymi elementami polskich wersji Wikipedii i~Wikisłownika. Wysoki stopień uporządkowania haseł w~Wikisłowniku daje nadzieję na wprowadzenie ulepszeń --- główną potencjalną barierą może być więc społeczność, która stara się podejmować decyzje konsensualnie.

Dobrym sposobem przedstawienia społeczności Wikisłownika jest przeciwstawienie jej grupie osób edytujących w~Wikipedii\footnote{Od tej pory pojęcia \emph{Wikipedia} i~\emph{Wikisłownik} domyślnie oznaczać będą wersje polskojęzyczne.}. W~tabeli~\ref{tab:wiki-wikt2} przedstawiono najbardziej widoczne różnice między nimi, które można zauważyć już po kilkugodzinnej lekturze archiwalnych dyskusji i~historii edycji.

\begin{table}[h]
\begin{center}
	\begin{tabularx}{\textwidth}{ XX }
		\toprule \textbf{Wikipedia} & \textbf{Wikisłownik} \\
		\midrule Dużo wandalizmów (edycji wykonywanych w~złej wierze)
			& Bardzo mało wandalizmów \\
		\midrule Kilkuset stale aktywnych redaktorów
			& Kilkunastu stale aktywnych redaktorów \\
		\midrule Wielu użytkowników mających kłopoty z~dostosowaniem się do zasad
			& Niewielu użytkowników mających kłopoty z~dostosowaniem się do zasad \\
		\midrule Liczne wojny edycyjne (wzajemne cofanie swoich edycji przez co najmniej dwóch redaktorów)
			& Praktycznie brak wojen edycyjnych \\
		\midrule Niezliczone strony archiwalnych dyskusji
			& Łatwy dostęp do archiwalnych dyskusji \\
		\midrule Dobra dokumentacja i~pomoc
			& Dobra pomoc, ale szczegółowa dokumentacja bardzo uboga \\
		\midrule Bardzo zróżnicowane obszary zainteresowania
			& Społeczność o~wspólnych zainteresowaniach lingwistycznych \\
		\bottomrule
	\end{tabularx}
\caption
	[Porównanie Wikipedii i~Wikisłownika --- aspekty społecznościowe]
	{Porównanie Wikipedii i~Wikisłownika --- aspekty społecznościowe. Zob. też tabelę~\ref{tab:wiki-wikt}}
\label{tab:wiki-wikt2}
\end{center}
\end{table}
Można uznać, że w~przypadku polskich wersji językowych Wikisłownik jest projektem budzącym znacznie mniejsze emocje niż Wikipedia, której powszechność powoduje nieuniknione problemy, takie jak zaangażowanie osób chcących wykorzystać projekt do celów marketingowych lub ideologicznych czy liczne wandalizmy w~wykonaniu znudzonych nastolatków. Społeczność redaktorów słownika zdecydowanie skupiona jest na stałym udoskonalaniu projektu. Z~racji jego rozmiaru łatwiej jest o~konsensus w~dyskusjach, a~dzięki mniejszej popularności innowacyjne rozwiązania techniczne mają większą szansę na realizację.

W~polskim Wikisłowniku działa 26~administratorów, spośród których 14 jest aktywnych. Kilku z~nich obsługuje swoje własne boty~\cite{wikt:admin}. Dyskusje na ogólne tematy toczą się w~tzw. \emph{Barze} (odpowiedniku \emph{Kawiarenki} w~Wikipedii). Na podstawie strony specjalnej \emph{Ostatnie zmiany} można oszacować poziom aktywności redaktorów w~stosunku do Wikipedii. Podczas gdy tam ostatnie 500~zmian obejmuje nieco ponad godzinę, w~Wikisłowniku jest to więcej niż doba. Charakterystyczny jest dużo większy niż w~Wikipedii udział nowo tworzonych haseł w~stosunku do poprawek w~starszych artykułach. To dość naturalne --- hasła słownikowe są prostsze i~często nie wymagają dalszych poprawek technicznych po ich utworzeniu. Dużą część haseł tworzy mała, najbardziej aktywna liczba użytkowników: dłuższa obserwacja pozwala na stwierdzenie, że proporcje są mniej więcej zgodne z~zasadą Pareto (\emph{20\% obiektów jest związanych z~80\% zasobów}). Wynika z~tego, że warto stworzyć aplikację, która będzie służyć zarówno nowym użytkownikom (czego skutkiem będzie zwiększenie liczby zaangażowanych osób), jak i~tym doświadczonym, bardziej aktywnym (większa wydajność i~satysfakcja z~użytkowania).


\section{Analiza wymagań}
Aby sprecyzować elementy, które złożą się na implementację aplikacji, konieczne jest przeprowadzenie analizy wymagań. Ten proces pozwoli na zrozumienie dokładnych potrzeb, które powinny zostać spełnione, i~rozbicie ich na konkretne, jasno zdefiniowane wymagania~\cite{guidebook}. Klasyczne metody powinny być jednak nieco zmodyfikowane ze względu na specyfikę projektu --- odbiorcą nie jest klient biznesowy, ale społeczność skupiona wokół otwartego projektu (zob.~podrozdział~\ref{sec:spec}).

W~przypadku niniejszego projektu specyfikowanie wymagań prowadzone było w~ramach dyskusji w~\emph{Barze} na łamach Wikisłownika oraz prywatnych rozmów z~redaktorami. Dodatkowo swoje opinie przekazało kilka osób niezwiązanych z~projektem, mogących spojrzeć na zagadnienie z~punktu widzenia nowego użytkownika.

Najważniejszym produktem, będącym wynikiem analizy wymagań, jest specyfikacja wymagań biznesowych. Zgodnie z~klasycznym podziałem zostały one podzielone na wymagania funkcjonalne i~niefunkcjonalne.

\subsection{Definicje}

\begin{itemize}
\item \textbf{Użytkownik} --- osoba dokonująca edycji w~Wikisłowniku.
\item \textbf{Użytkownik zalogowany} --- osoba zalogowana w~Wikisłowniku i~dokonująca w~tym projekcie edycji.
\item \textbf{Użytkownik niezalogowany} --- osoba niezalogowana w~Wikisłowniku i~dokonująca w~tym projekcie edycji. Takie edycje zostaną zapisane wraz z~adresem IP użytkownika, co umożliwia częściową identyfikację autora.
\item \textbf{Przestrzeń nazw} --- grupa stron w~Wikisłowniku o~wspólnym prefiksie zakończonym dwukropkiem. Przykładowe przestrzenie nazw to \kod|Szablon:|, \kod|Dyskusja:|. Szczególnym przypadkiem jest tzw. przestrzeń główna, skupiająca strony bez dodatkowego prefiksu --- w~niej znajdują się wszystkie hasła słownika.
\item \textbf{Hasło} --- strona w~przestrzeni głównej Wikisłownika.
\item \textbf{Stary formularz} --- dotychczasowa metoda wprowadzania danych, opisana w~podrozdziale~\ref{wikt:structure}.
\item \textbf{Nowy formularz} --- aplikacja będąca przedmiotem niniejszej pracy.
\item \textbf{Sekcja} --- odcinek pojedynczego hasła dotyczący użycia słowa w~pojedynczym języku, wyróżniony nagłówkiem drugiego stopnia (\kod|==|).
\item \textbf{Podsekcja} --- element sekcji hasła zgodny z~ogólną strukturą haseł (p.~dodatek~\ref{wikt:subsections}).
\item \textbf{Interwiki} --- linki pomiędzy poszczególnymi wersjami językowymi Wikisłownika.
\end{itemize}

\subsection{Wymagania funkcjonalne}
\subsubsection{Edycja istniejących haseł i~sekcji za pomocą nowego formularza}
Użytkownik musi mieć możliwość wprowadzenia zmian w~obrębie całego hasła Wikisłownika za pomocą nowego formularza. Formularz powinien umożliwiać edycję dowolnego elementu w~haśle --- w~szczególności dotyczy to sekcji wstępnej, zawierającej dane ogólne, niezwiązane z~żadnym językiem. Edycja ma polegać na uzupełnieniu wartości dla poszczególnych kluczy w~formularzu. Kluczami są tytuły kolejnych podsekcji, wartościami --- ich zawartość.

\subsubsection{Dodanie nowej sekcji do istniejącego hasła za pomocą nowego formularza}
Użytkownik musi mieć możliwość rozszerzenia istniejącego hasła o~kolejną sekcję językową. Sekcja powinna zostać automatycznie uzupełniona zawartością charakterystyczną dla wybranego języka.

\subsubsection{Utworzenie nowego hasła za pomocą nowego formularza}
Użytkownik musi mieć możliwość stworzenia nowego hasła, składającego się z~jednej lub wielu sekcji językowych. Każda sekcja powinna być utworzona w~sposób analogiczny do dodawania nowej sekcji do istniejącego hasła.

\subsubsection{Edycja i~utworzenie nowego hasła za pomocą starego formularza}
Użytkownik, który chce korzystać nadal ze starego formularza, musi mieć taką możliwość. Decyzja o~wyborze starego lub nowego formularza musi być łatwa do podjęcia i~możliwa do odwrócenia w~każdej chwili, także w~trakcie edycji lub tworzenia hasła.

\subsubsection{Wprowadzanie znaków specjalnych}
Użytkownik musi mieć możliwość prostego wprowadzania do formularza znaków specjalnych, w~szczególności liter alfabetów używanych w~językach opisywanych przez Wikisłownik.

\subsubsection{Usunięcie sekcji językowej}
Użytkownik musi mieć możliwość usunięcia dowolnej sekcji językowej z~hasła.

\subsubsection{Edycja tytułu sekcji językowej}
Użytkownik musi mieć możliwość zmiany tytułu sekcji językowej. Jest to konieczne z~powodu odrębnego traktowania haseł opisujących przysłowia, związki frazeologiczne i~inne wyrażenia składające się z~kilku wyrazów.

\subsubsection{Automatyczne pobieranie danych z~innych wersji językowych}
Użytkownik musi mieć możliwość wyszukania w~innych wersjach językowych Wikisłownika danych, które mogą być cennym uzupełnieniem edytowanego hasła. W~szczególności dotyczy to ilustracji, wymowy w~międzynarodowym alfabecie fonetycznym (IPA), nagrań dźwiękowych.%TODO UPDATE

\subsubsection{Automatyczne uzupełnianie podsekcji}
Aplikacja powinna automatycznie uzupełnić te spośród podsekcji w~danym haśle, które to umożliwiają. W~szczególności dotyczy to linków interwiki w~nowo dodawanej sekcji wstępnej, szablonów używanych do transliteracji, znacznika \kod|<references/>| w~podsekcji \emph{źródła}.%TODO UPDATE

\subsection{Wymagania niefunkcjonalne}
\subsubsection{Licencjonowanie}
Aplikacja musi zostać udostępniona na licencjach GNU~FDL~1.2 i~CC\dywiz{}BY\dywiz{}SA~3.0 --- wymagają tego zasady Wikisłownika i~wszystkich projektów Wikimedia. W~związku z~tym cały kod użyty w~aplikacji musi być dostępny już wcześniej na odpowiednich licencjach bądź udostępniony na nich przez autora pracy. Grafiki użyte w~interfejsie użytkownika muszą pochodzić z~serwisu Wikimedia Commons, zbierającego pliki na wolnych licencjach.

\subsubsection{Integracja z~Wikisłownikiem}
Aplikacja musi być w~pełni zintegrowana z~istniejącym interfejsem Wikisłownika. Z~tego powodu jedyną możliwą opcją jest aplikacja kliencka napisana w~języku JavaScript, w~rachubę nie wchodzi natomiast aplikacja desktopowa wymagająca dodatkowych czynności. Korzystanie z~nowego formularza powinno być inicjowane w~dokładnie ten sam sposób, co korzystanie ze starego formularza.

\subsubsection{Pomoc kontekstowa}
Nowy formularz musi zawierać pomoc kontekstową dla wszystkich elementów, których działanie nie jest oczywiste. Wyświetlanie pomocy nie może utrudniać edycji hasła.

\subsubsection{Interfejs użytkownika}
Interfejs użytkownika musi być prosty, czytelny i~intuicyjny. W~szczególności bardzo istotne jest, aby nowy formularz stanowił duże ułatwienie dla niedoświadczonego użytkownika i~zmniejszał barierę wejścia do projektu.

\subsubsection{Obsługiwane oprogramowanie}
Ponieważ Wikisłownik musi być dostępny do edycji dla jak największej liczby użytkowników, jest bardzo ważne, aby aplikacja obsługiwała wszystkie popularne przeglądarki internetowe: Firefox $\geq$~4, Chrome 13 (dzięki automatycznej aktualizacji w~przypadku tej przeglądarki nie występuje problem obsługi starszych wersji), Opera $\geq$~10 i~Internet Explorer $\geq$~7. W~przypadku braku wyłączonej obsługi JavaScriptu zachowanie formularza ma zostać niezmienione: załaduje się stary formularz z~wyłączonymi niektórymi funkcjami.

\subsubsection{Wydajność}
Aplikacja musi umożliwiać płynną pracę --- skrypty nie mogą być znacząco wolniejsze od skryptów użytych w~starym formularzu.

\subsubsection{Łatwość modyfikacji}
Ze względu na niestały charakter zasad panujących w~Wikisłowniku kod aplikacji musi być dostępny publicznie i~możliwy do modyfikacji w~każdej chwili. W~szczególności dotyczy to komunikatów dla użytkownika i~używanych stałych, które powinny być zebrane w~jednym miejscu.

\section{Specyfika tworzenia aplikacji dla wikispołeczności}
\label{sec:spec}
Opisywana aplikacja jest projektem dość szczególnym: nie powstaje na zamówienie klienta biznesowego, który mógłby określić dokładne wymagania. Ma także mało wspólnego z~klasycznymi projektami \emph{open source} --- z~założenia jej autorem jest jedna osoba, autor niniejszej pracy. Projekt formularza przyjmował coraz bardziej sprecyzowany kształt dzięki wypowiedziom członków otwartej społeczności i~ich wzajemnej dyskusji, duże znaczenie miały też testy wykonywane przez ochotników z~Wikisłownika. Aplikacja powstawała inkrementalnie, co kilka dni--tygodni skrypt na stronach Wikisłownika był aktualizowany do kolejnej działającej wersji z~nowymi funkcjami.

Dzięki takiemu modelowi budowy aplikacji na bieżąco wykrywano błędy w~działaniu, o~które było łatwo ze względu na poziom skomplikowania kluczowych modułów, takich jak parser wikitekstu. Projekt okazał się ciekawym doświadczeniem łączącym pozytywne cechy tworzenia oprogramowania dla klienta biznesowego i~programu na swój własny użytek.

Zarys projektu powstał jesienią 2010~roku w~wyniku pierwszej dyskusji w~\emph{Barze} Wikisłownika. Została wówczas założona podstrona poświęcona dyskusji wyłącznie nad tym zagadnieniem --- rozmowy w~ramach \emph{Baru} byłyby raczej niewygodne. Główny etap rozwoju aplikacji przypadł na okres od maja do września 2011~roku. W~toku dyskusji kilkukrotnie zmieniały się wymagania klienta, jakim jest wikispołeczność. W~tym przypadku, w~odróżnieniu od projektów biznesowych, cecha ta okazała się pozytywna. Dzięki bieżącym konsultacjom końcowy kształt aplikacji był satysfakcjonujący dla wszystkich stron uczestniczących w~przedsięwzięciu. Podrozdział~\ref{sec:impl-deploy} jest poświęcony ostatnim etapom projektu: wdrożeniu w~Wikisłowniku i~widokom na dalszy rozwój.
