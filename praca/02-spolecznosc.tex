Tematem tego rozdziału są społecznościowe aspekty zarówno samego Wikisłownika, jak i~niniejszej pracy magisterskiej. Na początku omówiono podstawowe założenia koncepcji \emph{wiki}, na której opiera się Wikisłownik. Kolejna część rozdziału to krótka analiza społeczności rozwijającej projekt. Na dalszych stronach znalazły się treści ściśle związane z~tworzoną aplikacją: opis rozwoju oprogramowania dla specyficznego klienta, jakim jest szersza społeczność, oraz szczegółowa analiza jego wymagań.

\section{Koncepcja \emph{wiki}}
Pojęcie \emph{wiki} określa witrynę internetową umożliwiającą tworzenie i~edycję dowolnej liczby połączonych ze sobą hiperłączami stron przez przeglądarkę internetową, za pomocą uproszczonego języka znaczników lub edytora WYSIWYG \cite{britannica}. Opisane w~rozdziale~\ref{chap:wikt} projekty Fundacji Wikimedia, szczególnie Wikipedia, są z~pewnością najbardziej znanymi i~największymi przykładami wiki.

Wiki to nie tylko technologia, ale także, a~może przede wszystkim, społeczność. Rozmiary wiki rozciągają się od małych, zamkniętych witryn firmowych do największej Wikipedii. To, co łączy wszystkie tego typu projekty, to otwarty dostęp do edycji i~tworzenia stron w~obrębie danej grupy. W~przypadku projektów Fundacji grupę tę stanowi niemal cała ludność świata dysponująca dostępem do internetu -- poza nielicznymi krajami, w~których sieć jest cenzurowana. Wśród redaktorów Wikipedii czy Wikisłownika nie istnieje żadna skodyfikowana hierarchia, co oznacza, że niezalogowany nowicjusz może zmienić stronę dokładnie tak samo jak administrator projektu. Założenie to od lat stanowi główny zarzut pod adresem Wikipedii -- negatywne opinie o~tej encyklopedii są szeroko rozpowszechnione~\cite{knol}. Mimo tego projekty stale rozwijają się i~podnoszą swoją jakość -- już w~2005~roku w~czasopiśmie ,,Nature'' opublikowano kontrowersyjne porównanie Wikipedii z~\emph{Encyclopædia Britannica}, z~którego wynikało, że obie encyklopedie stoją na podobnym poziomie~\cite{nature:britannica}. Jasno widać, że znaczenie projektów Fundacji Wikimedia jest dziś bardzo duże.

W~tworzeniu Wikipedii i~Wikisłownika biorą udział przedstawiciele całego społeczeństwa -- w~pracach uczestniczą zarówno profesorowie, jak i~studenci. Jak opisano w~sekcji~\ref{wikt:drawbacks}, dla wielu z~nich barierą jest poziom skomplikowania oprogramowania MediaWiki. Podczas gdy wielu uczestników, najczęściej tych o~zainteresowaniach ścisłych, nie ma problemów z~opanowaniem techniki edycji haseł, istnieje duża grupa osób, które deklarują, że mogłyby zaangażować się bardziej, gdyby nie konieczność opanowania złożonych sposobów wprowadzania informacji.

\section{Społeczność polskiej edycji Wikisłownika}
\label{sec:plsoc}


\begin{table}[h]
\begin{center}
	\begin{tabularx}{\textwidth}{ XX }
		\toprule \textbf{Wikipedia} & \textbf{Wikisłownik} \\
		\midrule Dużo wandalizmów (edycji wykonywanych w~złej wierze)
			& Bardzo mało wandalizmów \\
		\midrule Kilkuset stale aktywnych redaktorów
			& Kilkunastu stale aktywnych redaktorów \\
		\midrule Wielu użytkowników mających kłopoty z~dostosowaniem się do zasad
			& Niewielu użytkowników mających kłopoty z~dostosowaniem się do zasad \\
		\midrule Liczne wojny edycyjne (wzajemne cofanie swoich edycji przez co najmniej dwóch redaktorów)
			& Praktycznie brak wojen edycyjnych \\
		\bottomrule
	\end{tabularx}
\caption
	[Porównanie Wikipedii i~Wikisłownika -- aspekty społecznościowe]
	{Porównanie Wikipedii i~Wikisłownika -- aspekty społecznościowe. Zob. też tabelę~\ref{tab:wiki-wikt}}
\label{tab:wiki-wikt2}
\end{center}
\end{table}

Można uznać, że w~przypadku polskich wersji językowych Wikisłownik jest projektem znacznie spokojniejszym niż Wikipedia, której powszechność powoduje nieuniknione problemy, takie jak zaangażowanie osób chcących wykorzystać projekt do celów marketingowych lub ideologicznych czy liczne wandalizmy w~wykonaniu znudzonych nastolatków. Społeczność redaktorów słownika zdecydowanie skupiona jest na stałym udoskonalaniu projektu. Z~racji jego rozmiaru łatwiej jest o~konsensus w~dyskusjach, a~dzięki mniejszej popularności innowacyjne rozwiązania techniczne mają większą szansę na realizację.

\section{Analiza wymagań}

\section{Specyfika tworzenia aplikacji dla wikispołeczności}

