Celem niniejszej pracy było przedstawienie aplikacji, która ma za zadanie znacząco uprościć edycję haseł w~Wikisłowniku. Wydaje się, że w~bliskiej przyszłości potwierdzi się, że projekt ten zakończył się sukcesem. Wprawdzie często w~projektach Fundacji Wikimedia nowe funkcje wprowadzane są bardzo powoli, jednak, jak wielokrotnie wspomniano, specyfika  polskiego Wikisłownika odróżnia go od innych przedsięwzięć takich jak Wikipedia. Opisana aplikacja jest całkowicie niezależna od innych rozwiązań technologicznych przeznaczonych dla serwisów opartych na MediaWiki, a~jej przyszłość zależy praktycznie wyłącznie od stosunkowo jednomyślnej społeczności, traktującej nowy formularz edycyjny z~zadowoleniem, a~nawet entuzjazmem.

Co ważne, nowy formularz powstał praktycznie od zera, co było możliwe dzięki efektywnemu wykorzystaniu jQuery --- bez użycia tego typu biblioteki stworzenie zaawansowanego programu byłoby znacznie bardziej złożone. W~trakcie implementacji konieczne było rozwiązanie wielu problemów, które oczywiście nie mogły wszystkie zostać opisane w~niniejszej pracy dość wyczerpująco. Wprowadzenie aplikacji jako standardu w~Wikisłowniku powinno pozytywnie wpłynąć na rozwój tego projektu. Funkcje przetwarzające wikitekst zostały starannie przygotowane i~przetestowane we współpracy z~aktywnymi redaktorami, co pozwoli na dalsze ujednolicenie kodu artykułów --- na czym z~pewnością skorzystają kolejne projekty programistyczne w~Wikisłowniku.

Tworzenie tego typu aplikacji było ciekawym wyzwaniem, odmiennym od znakomitej większości projektów informatycznych. Dzięki otwartemu charakterowi aplikacja ma szansę okazać się przydatna dla tysięcy osób, które chciałyby wspomóc jeden z~najlepiej rozwiniętych polskich projektów opartych na koncepcji wiki. Kolejne miesiące prawdopodobnie przyniosą dalszy rozwój aplikacji --- podstawowy etap został zakończony, ale widoki na przyszłość prezentują się ciekawie. Szczególnie warte uwagi jest, że można liczyć na kolejne zastosowania modułu umożliwiającego wygodną automatyzację edycji na bazie API. W~polskich projektach Wikimedia jest to pierwsza aplikacja, która intensywnie korzysta z~tego interfejsu i~w~takim stopniu pozwala na uproszczenie procesu edycji artykułów, niekiedy rzeczywiście żmudnego. Skorzystają na tym zarówno doświadczeni użytkownicy, mający na koncie tysiące stworzonych haseł, jak i~nowicjusze: nie przerazi ich już widok, który ujrzą po kliknięciu w~zakładkę ,,edytuj''.
