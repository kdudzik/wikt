\usepackage{polski}
\usepackage{indentfirst}
\usepackage{parskip}
\usepackage{graphicx}
\usepackage{upquote}
\usepackage{tabularx}
\usepackage{setspace}
\usepackage{listings}
\usepackage{hyphenat}
\usepackage{mdwlist}
\usepackage{fontspec}
\usepackage{xunicode}
\usepackage{xltxtra}
\usepackage{xeCJK}
\usepackage{url}
\usepackage{booktabs}

\hyphenation{MySQL jQuery}

\defaultfontfeatures{Scale=MatchLowercase}
\setmainfont[Mapping=tex-text]{Gentium}
\setsansfont[Mapping=tex-text]{Calibri}
\setmonofont[Scale=0.7]{DejaVu Sans Mono}
\setCJKmainfont{Sazanami Mincho}

\author{Krzysztof Dudzik}
\nralbumu{248349}
\title{Aplikacja wspomagająca tworzenie i~edycję haseł w~polskim Wikisłowniku}
\tytulang{An application supporting article creation and edition for the Polish Wiktionary}
\kierunek{Informatyka}
\opiekun{dr. hab. Jerzego Tyszkiewicza, prof. UW\\Instytut Informatyki\\}
\date{2011}
\dziedzina{11.3 Informatyka\\}
\klasyfikacja{D. Software\\D.2. Software Engineering\\D.2.10. Design}
\keywords{Wikisłownik, Fundacja Wikimedia, MediaWiki, wiki, edytor, API, JavaScript, jQuery, interfejs użytkownika, społeczność internetowa}

% Tu jest dobre miejsce na Twoje własne makra i~środowiska:
\setkeys{Gin}{width=0.9\textwidth}
\graphicspath{{./screeny/}}
\setlength{\fboxsep}{0pt}
\setlength{\fboxrule}{0.2pt}
\setlength{\parskip}{1.2ex plus 0.5ex minus 0.2ex}
\setlength{\parindent}{5ex}
\frenchspacing
\brokenpenalty=1000
\clubpenalty=1000
\widowpenalty=1000
\renewcommand*{\figurename}{Ilustracja}
\newcommand{\spacer}{
	\begin{center}
		\textasteriskcentered
	\end{center}
}
\newcommand{\solidrule}{
	\begin{center}
		\line(1,0){250}
	\end{center}
}
\newenvironment{illustration}[0]{
	\begin{figure}[ht]
	\begin{center}
}{
	\end{center}
	\end{figure}
}
\newenvironment{description-sub}[0]{
	\begin{basedescript}{\desclabelstyle{\pushlabel}\desclabelwidth{6em}}\setlength{\itemsep}{-2mm}
}{
	\end{basedescript}
}
