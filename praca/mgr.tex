\documentclass{pracamgr}
\usepackage{polski}
\usepackage{indentfirst}
\usepackage{parskip}
\usepackage{graphicx}
\usepackage{upquote}
\usepackage{tabularx}
\usepackage{setspace}
\usepackage[utf8]{inputenc}

\author{Krzysztof Dudzik}
\nralbumu{248349}
\title{Aplikacja wspomagająca tworzenie i~edycję haseł w~polskim Wikisłowniku}
\tytulang{An application supporting article creation and edition for the Polish Wiktionary}
\kierunek{Informatyka}
\opiekun{dr. hab. Jerzego Tyszkiewicza, prof. UW\\Instytut Informatyki\\}
\date{Czerwiec 2011}
\dziedzina{11.3 Informatyka\\}
\klasyfikacja{D. Software\\D.2. Software Engineering\\D.2.10. Design}
\keywords{Wikisłownik, Fundacja Wikimedia, MediaWiki, wiki, edytor, API, JavaScript, jQuery, interfejs użytkownika, społeczność internetowa}

% Tu jest dobre miejsce na Twoje własne makra i~środowiska:
\newenvironment{illustration}[0]{
	\begin{figure}[h]
	\begin{center}
}{
	\end{center}
	\end{figure}
}
\setkeys{Gin}{width=0.9\textwidth}
\graphicspath{{./screeny/}}
\setlength{\fboxsep}{0pt}
\setlength{\fboxrule}{0.2pt}
\setlength{\parskip}{1.2ex plus 0.5ex minus 0.2ex}
\setlength{\parindent}{5ex}
\frenchspacing
\renewcommand*{\figurename}{Ilustracja}

% koniec definicji

\begin{document}
\maketitle

\begin{abstract}
  Tematem pracy jest aplikacja służąca do ułatwienia pracy autorów haseł w~polskim Wikisłowniku. Jej funkcje mają w~maksymalny możliwy sposób ułatwić tworzenie i~edytowanie haseł osobom bez wiedzy informatycznej i~technicznej, a~także automatyzować możliwie wiele rutynowych czynności wykonywanych przy redagowaniu hasła, jak tworzenie łącz do haseł powiązanych, zautomatyzowane szukanie przykładów użycia, wystąpień w~związkach frazeologicznych, wyrazów bliskoznacznych, innych słów, którą formę gramatyczną mogłoby stanowić hasło~itp. Dodatkowo aplikacja może przejąć część funkcji realizowanych obecnie za~pomocą botów.
\end{abstract}

\tableofcontents

\chapter{Wprowadzenie}
Żyjemy w~czasach, w~których nieustannie zmienia się sposób wyszukiwania informacji przez przeciętnego człowieka. Z~roku na rok coraz mniejszą rolę odgrywają papierowe kompendia takie jak encyklopedie i~słowniki, stopniowo przybierają natomiast na znaczeniu elektroniczne bazy wiedzy -- szczególnie zaś internetowe zbiory danych. Przyczyny tego stanu rzeczy są oczywiste: chodzi przede wszystkim o~wygodę korzystania ze~stron internetowych. Brak możliwości wyszukiwania w~obrębie ogromnych ilości danych powoduje, że encyklopedie i~słowniki w~postaci książek stają się o~wiele mniej atrakcyjne dla kogoś, kto chce zdobyć nowe informacje.

Wszechobecny dostęp do internetu sprawia, że to właśnie w~sieci WWW powstają najbardziej popularne bazy ludzkie wiedzy. Nie ma chyba internautów, którzy nie korzystaliby, rzadziej lub częściej, z~Wikipedii -- internetowej encyklopedii pisanej przez ochotników. Właśnie fakt, że encyklopedia ta współtworzona jest przez amatorów, stanowi o~jej wyjątkowym charakterze, który zostanie w~tej pracy pokrótce opisany. Wikipedia stale utrzymuje się w~pierwszej dziesiątce najczęściej odwiedzanych stron, a~pod wieloma względami jest to dziś najlepsza istniejąca encyklopedia. Przed kilkoma laty głośne było porównanie jej z~prestiżową \emph{Encyclopaedia Britannica} -- okazało się, że różnice w~poziomie merytorycznym są niewielkie.

O~ile przewrót w~kategorii encyklopedii właściwie już się dokonał, nieco inaczej wygląda rywalizacja słowników. Oczywiście wyraźnie widać, że i~tu papierowe edycje są coraz mniej popularne. Różnice uwidaczniają się, gdy przeanalizowana zostanie sytuacja słowników internetowych. Tak zwany siostrzany projekt Wikipedii, Wikisłownik, nie dominuje wśród konkurencji -- zarówno na świecie, jak i~w~Polsce. Przyczyny tego stanu rzeczy są złożone. Autor postanowił skupić się na kilku zagadnieniach, uwidaczniających się w~polskojęzycznej wersji Wikisłownika. W~tym celu konieczne było zbadanie społeczności zaangażowanej w~tworzenie tego projektu. Jego efektem było wykonanie prac programistycznych, których opis stanowi główną część niniejszego opracowania.

W~przypadku wszystkich projektów opartych na silniku programistycznym MediaWiki istotną barierą rozwoju jest sama technologia. Każdy ochotnik ma możliwość uczestniczenia w~rozwoju portalu, wiąże się to jednak z~koniecznością przystosowania się do wymagań stawianych przez oprogramowanie. Edytowanie haseł w~internetowej encyklopedii czy słowniku jest praktycznie niemożliwe dla osoby bez wcześniejszego przygotowania lub znacznej wiedzy techniczno-informatycznej. Oprogramowanie MediaWiki oparte jest bowiem na tzw. wikikodzie (także: wikitekst, wikiskładnia), czyli języku opisu struktury i~wyglądu strony internetowej -- prostszym niż HTML, jednak wciąż nieintuicyjnym dla kogoś, kto nie miał wcześniej do czynienia z~tego typu edytorami. Dlatego wielu potencjalnych współautorów zniechęca się do projektu już przy pierwszej próbie poprawy artykułu.

Aby zmienić tę sytuację, przygotowany został nowy edytor, dostosowany specjalnie do potrzeb polskiego Wikisłownika. Aplikacja pozwala na o~wiele prostsze tworzenie nowych i~zmienianie starych haseł niż poprzednia, standardowa. Dzięki użyciu jej jako domyślnej w~projekcie popularyzacja edytowania Wikisłownika wśród fachowców w~dziedzinach lingwistycznych okaże się łatwiejsze -- zniknie podstawowa bariera, jaką jest konieczność dostosowania się do skomplikowanych technicznych wymagań stawianych przez użyte oprogramowanie. Dodatkowo nowa aplikacja umożliwia zaawansowaną automatyzację tworzenia hasła. Wiele z~czynności zintegrowanych z~nowym edytorem do tej pory wymagało mozolnych poszukiwań w~artykułach Wikisłownika oraz innych projektach. Dzięki użyciu API udostępnianego przez serwisy Fundacji Wikimedia skomplikowane przeszukiwanie tysięcy stron udało się sprowadzić do kilku kliknięć.

W~dalszej części pracy opisany został proces tworzenia tego edytora. Pierwszy rozdział charakteryzuje pokrótce sam Wikisłownik, jak i~pokrewne projekty oraz oprogramowanie w~nich użyte. Następnie opisano społecznościowe aspekty tworzenia tego typu aplikacji ze szczególnym uwzględnieniem koncepcji \emph{wiki}. Ostatni rozdział wyczerpująco przedstawia szczegóły projektowe i~implementacyjne aplikacji.

\chapter{Wikisłownik}
Rozdział ten stanowi charakterystykę Wikisłownika -- sieciowego słownika opartego na oprogramowaniu MediaWiki. Wikisłownik jest jednym z~największych i~najpopularniejszych słowników dostępnych w~polskim internecie. W~kolejnych sekcjach projekt ten został opisany na różnych poziomach szczegółowości. Omówiono zarówno oprogramowanie, na jakim bazuje słownik, jak i~swego rodzaju ,,ekosystem'', w~którym znajduje on swoje miejsce.

\section{Projekty Fundacji Wikimedia}
Podmiotem odpowiedzialnym m.in. za rozwój Wikisłownika jest Wikimedia Foundation Inc.\ (opisywana dalej jako ,,Fundacja'') -- organizacja non-profit mająca siedzibę w~San Francisco w~Stanach Zjednoczonych, istniejąca od 2003 roku. Jak informuje strona internetowa polskiego partnera Fundacji, Stowarzyszenia Wikimedia Polska, \emph{celem fundacji jest sprzyjanie tworzeniu i~rozwojowi projektów o~otwartej treści opartych na technologii WikiWiki oraz dostarczanie społeczności internetowej pełnej zawartości wymienionych projektów za darmo i~bez zamieszczania reklam.} %cite
Doskonale znaną, sztandarową inicjatywą Fundacji jest Wikipedia (\texttt{http://www.wikipedia.org}) -- największa obecnie encyklopedia internetowa, dostępna w~281 językach (stan z~maja 2011 roku) i~zawierająca ponad 18~milionów haseł, w~tym ponad 3,6~miliona w~największej, angielskojęzycznej\footnote{Oficjalnie w~projektach Fundacji używane są określenia typu \emph{angielskojęzyczny}, \emph{polskojęzyczny}. Choć w~przypadku wersji polskojęzycznej znakomita większość uczestników projektów pochodzi z~Polski, nie jest to regułą dla innych edycji. W~dalszej części pracy przyjęto uproszczenie polegające na tym, że określenia typu \emph{polska Wikipedia}, \emph{angielski Wikisłownik} traktowane są jako tożsame z~określeniami używającymi sformułowań z~cząstką \emph{-języczny}.} edycji. Mimo częstej krytyki tego przedsięwzięcia faktem jest, że Wikipedia jest miejscem, z~którego miliony osób korzystają, by pozyskać informacje z~najróżniejszych dziedzin. Obecny stan rzeczy możliwy jest dzięki pracy wielkiej liczby wolontariuszy tworzących artykuły bez wynagrodzenia.

\begin{illustration}
	\fbox{\includegraphics{plwikipedia}}
	\caption{Polska edycja Wikipedii}
\end{illustration}

Wikipedia jest najbardziej znanym, ale nie jedynym projektem pod opieką Fundacji. Pozostałe to tzw.\ ,,projekty siostrzane'', w~szczególny sposób uwzględniane również przy tworzeniu haseł w~encyklopedii. Oto lista wspieranych przez Fundację wielojęzycznych inicjatyw:
\begin{itemize}
	\item Wikisłownik (ang. \emph{Wiktionary}) -- wielojęzyczny słownik internetowy, będący głównym przedmiotem niniejszej pracy,
	\item Wikicytaty (ang. \emph{Wikiquotes}) -- zbiór cytatów autorstwa znanych osób, z~filmów i~książek, przysłów i~porzekadeł,
	\item Wikibooks -- serwis z~,,otwartymi'' (opartymi na wolnej licencji) podręcznikami,
	\item Wikiźródła (ang. \emph{Wikisource}) -- zbiór dokumentów źródłowych w~wersjach oryginalnych i~tłumaczonych, nieograniczonych prawem autorskim,
	\item Wikinews -- otwarty serwis informacyjny,
	\item Wikimedia Commons -- repozytorium mediów (zdjęć, grafik, filmów) dostępnych na wolnej licencji, z~którego korzystają pozostałe projekty Wikimedia,
	\item Wikispecies -- katalog gatunków organizmów żywych,
	\item Wikiversity -- materiały edukacyjne i~naukowe,
	\item Wikimedia Incubator -- metaprojekt umożliwiający tworzenie nowych inicjatyw wspieranych przez Fundację.
	\item Meta-Wiki -- projekt ułatwiający koordynację wszystkich pozostałych.
\end{itemize}
Wszystkie projekty łączy sposób ich powstawania -- możliwość edycji dostępna jest praktycznie dla każdego internauty. Nie dotyczy to co~prawda kilku krajów, w~których projekty Fundacji zablokowane są w~ramach cenzury internetu, jednak ogromna większość osób dysponujących łączem internetowym ma szansę stać się jednymi spośród współautorów haseł.

Drugą cechą wspólną są wolne licencje, na których udostępniana jest zawartość wszystkich serwisów. Po reformie w~czerwcu 2009~roku treść Wikipedii i~projektów siostrzanych dostępna jest nie tylko na licencji GNU FDL (Free Documentation License), ale także na kompatybilnej z~nią CC-BY-SA 3.0 (Creative Commons Attribution-ShareAlike / Uznanie Autorstwa -- Na Tych Samych Warunkach). %cite http://meta.wikimedia.org/wiki/Licensing_update/Result
Oznacza to, że można ją dowolnie wykorzystywać we~własnych dziełach pod warunkiem podania oryginalnych autorów i~zachowania pierwotnej licencji.

\section{Oprogramowanie MediaWiki}
Sama działalność wolontarystyczna redaktorów projektów Wikimedia nie wystarczyłaby do stworzenia serwisów internetowych o~obecnych kształtach. Konieczne jest oczywiście również zapewnienie oprogramowania, które umożliwi płynną współpracę przy tworzeniu haseł. Tym oprogramowaniem jest wolna platforma MediaWiki tworzona zgodnie z~zasadami \emph{open source}. System MediaWiki napisany jest w~języku PHP i~opiera się na bazie danych (w~przypadku projektów Wikimedia jest to MySQL). Dla inicjatyw Wikimedia stanowi szkielet programistyczny od samego ich początku, a~od 2002~roku stale się rozwija. W~czerwcu 2011~roku wersją używaną w~projektach było MediaWiki~1.17.

System MediaWiki używany jest nie tylko w~projektach wspieranych przez Fundację, ale także w~tysiącach innych, mniejszych lub większych, co jest możliwe dzięki wysokiemu stopniowi konfigurowalności i~dużej liczbie dostępnych rozszerzeń. Są to w~dużej mierze serwisy o~podobnym charakterze, umożliwiające swobodną wymianę informacji na dowolny temat. MediaWiki bywa także używane w~firmowych intranetach i~wszędzie tam, gdzie zachodzi potrzeba udostępnienia materiałów do edycji dużej liczbie użytkowników.

\subsection{Edytowanie i~wikitekst}
Strony w~projektach opartych na platformie MediaWiki na~ogół nie mogą być czystym tekstem, pozbawionym formatowania. Przykładowo hasła w~encyklopedii muszą zachowywać określoną strukturę -- występuje więc podział na sekcje, ilustracje, różne rodzaje formatowania (kursywa, wytłuszczenie), przypisy czy powtarzalne fragmenty. Szczególnie istotnym elementem są linki pomiędzy poszczególnymi artykułami, wyróżniające projekty Fundacji na tle ich papierowych, ale też elektronicznych konkurentów. Odnośniki pozwalają błyskawicznie przemieszczać się między hasłami, by w~ten sposób uzyskiwać kolejne informacje wspomagające przyswajanie wiedzy.

Linki i~formatowanie na stronach internetowych tworzone są za pomocą elementów języka HTML lub XHTML. O~ile języki te są proste w~obsłudze dla specjalisty informatyka, to laik nie jest w~stanie tworzyć za ich pomocą stron bez uprzedniego dłuższego przygotowania. Aby umożliwić bezproblemową edycję stron internetowych osobom bez wykształcenia informatycznego, programiści MediaWiki zaprojektowali tzw.\ wikitekst -- uproszczony język opisu stron, pozwalający na realizację wymienionych elementów. Porównanie niektórych z~nich znajduje się w~tabeli \ref{html-wiki}.

\begin{table}[h]
\begin{center}
\label{html-wiki}
\footnotesize{
	\begin{tabularx}{\textwidth}{ |l|X|X| }
		\hline & Wikitekst & XHTML \\
		\hline
		\hline Kursywa & \texttt{''Tekst''} & \texttt{<em>Tekst</em>} \\
		\hline Wytłuszczenie & \texttt{'''Tekst'''} & \texttt{<strong>Tekst</strong>} \\
		\hline Nagłówek & \texttt{== Nagłówek ==} & \texttt{<h2>Nagłówek</h2>} \\
		 & \texttt{=== Nagłówek ===} & \texttt{<h3>Nagłówek</h3>} \\
		\hline Odnośnik wewnętrzny & \texttt{[[Strona]]} & \texttt{<a href="/wiki/Strona">Strona</a>} \\
		 & \texttt{[[Strona|strony]]} & \texttt{<a href="/wiki/Strona">strony</a>} \\
		\hline Odnośnik zewnętrzny & \texttt{[http://www.google.com Google]}
		 & \texttt{<a href="http://www.google.com">\newline Google</a>} \\
		\hline Obraz & \texttt{[[Plik:Przykład.png|thumb|Podpis]]} & \texttt{<img src="\dots /Przykład.png"/><br/>\newline <div class="caption">Podpis</div>} \\
		\hline Podział na akapity & \texttt{Pierwszy akapit \newline \newline Drugi akapit}
		 & \texttt{<p>Pierwszy akapit</p>\newline <p>Drugi akapit</p>} \\
		\hline Lista nienumerowana & \texttt{* Element\newline * Element\newline * Element}
		 & \texttt{<ul>\newline <li>Element</li>\newline <li>Element</li>\newline <li>Element</li>\newline </ul>}\\
		\hline Lista numerowana & \texttt{\# Element\newline \# Element\newline \# Element}
		 & \texttt{<ol>\newline <li>Element</li>\newline <li>Element</li>\newline <li>Element</li>\newline </ol>}\\
		\hline
	\end{tabularx}
}
\caption{Porównanie HTML i wikitekstu}
\end{center}
\end{table}

Łatwo można zauważyć, że używanie wikitekstu jest o~wiele prostsze niż nauka XHTML-a. Jeśli zachodzi potrzeba zaawansowanego formatowania, możliwe jest także użycie znaczników XHTML. W~przypadku standardowego formatowania jest to jednak niewskazane ze względu na dobro niedoświadczonych edytorów.

Bardzo istotnym elementem wikitekstu są szablony -- predefiniowane fragmenty kodu z~opcjonalnymi parametrami. Szablony można uznać za odpowiednik procedur/funkcji w~językach programowania. W~projektach opartych na MediaWiki szablony pełnią przede wszystkim dwie główne funkcje:
\begin{itemize}
	\item upraszczają kod -- pozwalają np.\ na zastąpienie skomplikowanego kodu XHTML (a~także jeszcze bardziej złożonych funkcji parsera MediaWiki) krótkim wywołaniem szablonu,
	\item standaryzują strony -- często wykorzystywane fragmenty wywoływane są zawsze w~dokładnie ten sam sposób.
\end{itemize}
We~wszystkich większych projektach Fundacji szablony są bardzo często wykorzystywane. Przy tym stanowią duże ułatwienie dla technicznie zaawansowanych autorów, którzy wspomagają się przy tworzeniu haseł dodatkowymi technologiami. Na używaniu szablonów korzystają przede wszystkim boty, czyli programy dokonujące edycji samodzielnie po uprzednim przygotowaniu lub pod stałą opieką programisty. W~większości projektów przyjęte jest, że każdy bot ma własne konto użytkownika, nie używa natomiast konta swojego ,,właściciela''. Dzięki botom możliwe jest np.\ masowe tworzenie haseł w~Wikipedii na ściśle określony temat, jeśli istnieje dobre źródło w~formie czytelnej dla komputera (takich jak opisy asteroid czy wszystkich miejscowości lub jednostek administracyjnych w~danym kraju). Innym ich zastosowaniem jest automatycznie uzupełnianie tzw.\ interwiki -- czyli odnośników pomiędzy poszczególnymi wersjami językowymi tego samego hasła.

Szablony mogą być wykorzystywane w~prosty sposób nawet przez początkujących użytkowników -- aby wywołać szablon, wystarczy wpisać jego nazwę
 pomiędzy podwójnymi nawiasami klamrowymi (\verb@{{Nazwa szablonu}}@ lub \verb@{{Nazwa szablonu|parametr=wartość}}@). W~polskim Wikisłowniku szablony pełnią szczególną rolę -- i~to m.in.\ dzięki ich szerokiemu zastosowaniu w~projekcie zrodził się pomysł na niniejszą pracę. Szczegóły tego zagadnienia zostaną przedstawione w~sekcji~\ref{sec:plwikt}.

\section{Wiktionary -- Wikisłownik}
Jednym z~największych projektów siostrzanych Wikipedii jest Wikisłownik, w~wersji angielskiej (i~wielu innych) noszący nazwę \emph{Wiktionary} (\texttt{http://www.wiktionary.org}). Ten słownik internetowy nie rozwinął się jeszcze tak prężnie jak encyklopedia, zwłaszcza jeśli chodzi o~polską edycję. Jest dziś jednak jednym z~największych słowników w~sieci, a~w~pewnych zastosowaniach stanowi najlepszy wybór. Dużą zaletą Wikisłownika jest jego wielojęzyczność -- w~tym samym serwisie znaleźć można hasła w~ponad~250~językach. W~przypadku niektórych z~nich jest to praktycznie jedyny słownik internetowy lub nawet jedyny dostępny słownik w~ogóle. Przykładem może być polski Wikisłownik, który zawiera prawdopodobnie jedyny polski słownik języka hawajskiego czy największe słowniki języków suahili i~jidysz. %cite http://pl.wiktionary.org/wiki/Wikis%C5%82ownik:Dlaczego_Wikis%C5%82ownik

Wspomniane zostało zastosowanie szablonów do standaryzacji kodu źródłowego i~struktury haseł. Trzeba jednak zaznaczyć, że dotyczy to wyłącznie haseł w~obrębie jednej wersji językowej Wikisłownika, są one bowiem niezależne od siebie i~od Fundacji. Wspólne dla wszystkich edycji jest jedynie oprogramowanie MediaWiki i~umiejscowienie na serwerach Fundacji pod adresem \texttt{xxx.wiktionary.org}, gdzie zamiast \texttt{xxx} wstawiany jest dwu- lub trzyliterowy skrót języka (np. \texttt{pl} = język polski, \texttt{de} = język niemiecki, \texttt{sq} = język albański, \texttt{csb} = język kaszubski). Wszystkie kwestie organizacyjne w~obrębie wersji językowej ustalane są w~ramach dyskusji i~głosowań przez internetową społeczność. Głosowania służą także wyborowi administratorów projektu, czyli użytkowników mających dodatkowe uprawnienia, spośród których najważniejsze to usuwanie i~zabezpieczanie haseł oraz blokowanie użytkowników działających na szkodę projektu. W~lipcu 2011~roku w~angielskim Wikisłowniku działało aktywnie 76~administratorów, %cite http://en.wiktionary.org/wiki/Wiktionary:Administrators
zaś w~polskim -- 14. %cite http://pl.wiktionary.org/wiki/Wikis%C5%82ownik:Administratorzy
Dla porównania Wikipedia w~języku angielskim ma 1541~administratorów (niekoniecznie aktywnych), wersja polska natomiast 163. %cite http://meta.wikimedia.org/wiki/Administrators_of_various_Wikipedias

\section{Polska edycja Wikisłownika}
\label{sec:plwikt}
\begin{illustration}
	\fbox{\includegraphics{plwikt}}
	\caption{Polska edycja Wikisłownika}
\end{illustration}
Aplikacja opisana w~niniejszej pracy przeznaczona jest dla polskiej edycji Wikisłownika. Konieczne jest zatem scharakteryzowanie tego projektu i~podkreślenie jego specyfiki. Ze~względu na odrębność poszczególnych wersji językowych prawdopodobnie w~innych nie będzie można wykorzystać wykorzystać aplikacji. Z~pewnością może ona być dla nich bardzo przydatna -- duża część kodu źródłowego może być używana niezależnie od reszty, a~przy tym jego funkcje są stosunkowo niezależne od specyfiki danej wersji.

\subsection{Struktura hasła}

\chapter{Aspekty społecznościowe}
\section{Koncepcja \emph{wiki}}
\section{Społeczność polskiej edycji Wikisłownika}
\section{Specyfika tworzenia aplikacji dla wikispołeczności}

\chapter{Opis implementacji}
\section{Wprowadzenie}
\section{Formularz edycyjny}
\section{Automatyzacja edycji hasła}
\section{Wdrożenie}

\chapter{Podsumowanie}


\appendix

\bibliographystyle{unsrt}
\addcontentsline{toc}{chapter}{Bibliografia}
%\sbibliography{mgr}{}
%\listoffigures
%\listoftables


\end{document}


%%% Local Variables:
%%% mode: latex
%%% TeX-master: t
%%% coding: latin-2
%%% End:
